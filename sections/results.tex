\section{Results}
\label{sec:results}

%% FIG: MODEL VS. WEIGHT OF L1 PENALTY (LASSO)
\begin{figure}
    \centering
    \includegraphics[width=\linewidth]{figures/lasso_ph.png}
    \caption{\Ram{This picture needs to appear later.} As the weight of the L1 penalty term increases, the Koopman operator matrix becomes more dense, and the model error decreases then levels off. This shows that there is a much sparser representation of the Koopman operator than the least-squares solution that generates a model of nearly identical accuracy.}
    \label{fig:lasso}
\end{figure}

%% FIG: Noise inherent to the system
\begin{figure}
    \centering
    \includegraphics[width=\linewidth]{figures/noise_ph.png}
    \caption{\Dan{Need to use different colors for y1 y2. Blue and red already used to show state space vs. lifted space.}}
    \label{fig:noise}
\end{figure}


%% FIGURE: Linear vs. Koopman model predictions
\begin{figure}
    \centering
    \includegraphics[width=\linewidth]{figures/predictionComparison_ph.png}
    \caption{\Dan{Placeholder for plot showing the prediction comparison between the koopman and the linear state space model.}}
    \label{fig:compare_blockM}
\end{figure}

%% FIGURE: Visual comparison of controller performance for the block M.
\begin{figure*}
    \centering
    \includegraphics[width=\linewidth]{figures/compare_blockM_300s_draft.png}
    \caption{The results of each controller to performing task 1. Reference trajectory only (left). Koopman MPC (middle). Linear MPC (right). Laser dot trajectory is shown in red, the reference trajectory is shown in blue.}
    \label{fig:compare_blockM}
\end{figure*}

%% FIGURE: Visual comparison of controller performance for the pacman.
\begin{figure*}
    \centering
    \includegraphics[width=\linewidth]{figures/compare_pacman68_90s_draft.png}
    \caption{The results of each controller to performing task 2. Reference trajectory only (left). Koopman MPC (middle). Linear MPC (right). Laser dot trajectory is shown in red, the reference trajectory is shown in blue.}
    \label{fig:compare_pacman}
\end{figure*}

%% TABLE: RMSE results table
\begin{table}[]
    \rowcolors{2}{white}{gray!25}
    \setlength\tabcolsep{5pt} % default value: 6pt
    \centering
    \caption{RMSE (cm) over all trajectory following tasks \Dan{Fill in real results later}}
    \begin{tabular}{|c|c|c|c|c|c|c|c|c|}
        \hline
        \rowcolor{white} 
        & \multicolumn{6}{c |}{\textbf{Task}} & & \textbf{Std.} \\
        \cline{2-7} \rowcolor{white}
        \multirow{-2}{*}{\textbf{Controller}} & $1$ & $2$ & $3$ & $4$ & $5$ & $6$ & \multirow{-2}{*}{\textbf{Avg.}} & \textbf{Dev.} \\
        \hline
        % RESULTS FOR ROBOT A
        Koopman MPC &  2.4  &  2.0  &  2.9  &  1.7  &  1.5  &  2.0 & 2.1 & 0.5 \\
        Linear MPC  &  5.8  &  4.0  &  6.6  &  3.9  &  2.8  &  3.5 & 4.5 & 1.5 \\
        Nonlinear MPC &  5.1  &  3.1  &  9.9  &  3.0  &  1.8  &  4.8 & 4.6 & 2.9 \\
        % Ham.-Weiner &  7.0  &  4.5  &  6.9  &  3.0  &  2.3  &  3.1 & 4.5 & 2.0 \\
        % \multirow{-5}{*}{\cellcolor{white} \rotatebox[origin=c]{90}{\textbf{Robot A}}}
        % NLARX       &  5.0  &  3.0 &  12.0  &  3.8  &  2.1  &  2.8 & 4.8 & 3.7 \\
        \hline
        % % RESULTS FOR ROBOT B
        % \cellcolor{white} & Koopman & & & & & & & & \\
        % \cellcolor{white} & Neural Net & & & & & & & & \\
        % \cellcolor{white} & State Space & & & & & & & & \\
        % \cellcolor{white} & Ham.-Weiner & & & & & & & & \\
        % \multirow{-5}{*}{\cellcolor{white} \rotatebox[origin=c]{90}{\textbf{Robot B}}}
        % & NLARX & & & & & & & & \\
        % \hline
    \end{tabular}
    \label{tab:RMSE}
\end{table}
