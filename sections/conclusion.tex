\section{Conclusion}
\label{sec:conclusion}

% Reiteration of contribution
In this work, a data-driven modeling and control method based on Koopman operator theory was successfully applied to a soft robot.
The Koopman-based MPC controller was shown to be capable of commanding a soft robot to follow a reference trajectory better than an MPC controller based on another data-driven model.
% Reiterarion of why important for soft robotics
By making explicit control-oriented models of soft robots easier to construct, this method holds the promise to unlock the untapped potential of soft robots by enabling the rapid development of new control strategies.

% Current shortcomings that should be addressed in future work (more dynamics excitement, higher dimensional systems)
While these preliminary results are promising, further work is needed to make such methods feasible for higher dimensional robotic systems.
Toward that end, this work introduced a method for promoting sparsity in matrix representations of the Koopman model.
Additional work will explore strategies for further promoting sparsity, choosing the most effective basis of observables, and building models that can account for external loading and contact forces.





% \Dan{For now this is just a list of talking points}
% Discussion of results:
% The model predictive controller using the the Koopman model outperformed the other controllers in a variety of trajectory following tasks.

% Sources of Error:
% -Model inaccuracies due to insufficient data.
% -Model inaccuracies due to Koopman truncation.
% -Poor performance of electronic pressure regulators.
% -Limited accuracy of camera based laser tracking system.

% Current Shortcomings of Method:
% -Curse of dimensionality, but sparsity could help (cite that paper)
% -Does not generalize outside of observed data, could be solved by switching controllers/hybrid models.

% Take Aways:
% -Has potential to revolutionize soft robot control by providing much more control friendly representation of dynamics.
% -Soft robots are well suited for a data-driven method because they can be observed safely under randomized control inputs.
% -This is the first time this method has been shown to be effective for controlling a real soft robotic system.